% Use pdflatex !!!!
\documentclass[dvipdfmx]{beamer}

\usepackage{cancel}
%\usepackage{CJK}
\usepackage{times}
\usepackage{pxjahyper}
\usepackage{minijs}%和文用
%\usepackage{hyperref}
\renewcommand{\kanjifamilydefault}{\gtdefault}%和文用に

\mode<all>{
  \usetheme{Taiwan}
}

\newcommand{\emphblue}[1]{\textcolor{blue}{#1}}
\renewcommand{\alert}[1]{\textbf{\textcolor{red}{#1}}}
\newcommand{\alertblue}[1]{\textbf{\textcolor{blue}{#1}}}
\newcommand{\hint}[2][]{
  \vskip1em
  \begin{columns}
    \begin{column}{8cm}
      \begin{yellowbox}{}
        #2
      \end{yellowbox}
    \end{column}
    #1
  \end{columns}
}
\newcommand{\chint}[2][]{\hint[#1]{\begin{center}#2\end{center}}}
%%%%%%%%%%%%%%%%%%%%%%%%%%%%%%%%%%%%%%%%%%%%%%%%%%%%%%%%


\title[タイトル]{量子情報理論}
\author{山田健太郎}
\institute[JPN]{XX大学}
\date{\today}

\begin{document}

\begin{frame}\frametitle{}
\titlepage %表紙
\end{frame}

\section*{TOC}
\begin{frame}\frametitle{Contents}
\tableofcontents %目次
\end{frame}

\section{楕円曲線暗号とは}
\subsection{This is subsection}
\begin{frame}\frametitle{量子情報理論}
$$
\frac{1}{1+e^{-x}}
$$

\begin{block}{量子情報}
    \begin{itemize}
         \item 量子鍵配送
         \item NP完全問題
         \item 量子もつれ
    \end{itemize}
\end{block}

\begin{alertblock}{BB84}
    \begin{itemize}
         \item C. BennettとG. Brassard によって再発見
         \item 実装されている量子鍵配送のほとんど
         \item 検出器(ボブ)のノイズにより傍受の検知が出来なくなる
    \end{itemize}
\end{alertblock}
\end{frame}

\begin{frame}\frametitle{量子もつれ}
\centering
\begin{eqnarray}
i\rho^{\prime}=U\left(  i\rho\right)  U^{\dagger}=Ad_{U}\left(  i\rho\right)\\
i\rho^{\prime}=Ad_{U}\left(  i\rho\right)  =U\left(  i\rho\right)  U^{-1}%
\end{eqnarray}
\vspace{16pt}
\href{http://www.csee.umbc.edu/~lomonaco/ams/lecturenotes/figssamtangle/unusedfigs/samtangle.tex}{ref: An Entangled Tale of Quantum Entanglement}
\end{frame}


\end{document}
